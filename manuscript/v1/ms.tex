\documentclass[iop,apj]{emulateapj}

%\documentclass{aastex}
\usepackage{natbib}
\usepackage{amsmath,amssymb}
\usepackage{../style_files/apjfonts}
\usepackage{xspace}
\usepackage{rotating}
%\usepackage[max]{morefloats}
\usepackage[usenames]{xcolor}


\newcommand{\todo}{\textcolor{red}}
\newcommand{\CiteNeed}{\textcolor{red}{\scriptsize[citation needed]}}
\newcommand{\CheckMe}{\textcolor{orange}{\scriptsize[check me]}}
\renewcommand{\check}{\textcolor{orange}}
\newcommand{\mv}{}
\newcommand{\je}{}
\newcommand{\st}[1]{\textcolor{blue}{SRT: #1}}
\newcommand{\rvh}{}
\newcommand{\cm}{}
\newcommand{\cmt}{\textbf}
\newcommand{\num}[2]{#1\times 10^{#2}} %scientific notation

\usepackage[backref, breaklinks, plainpages=false, colorlinks=true, anchorcolor=cyan, linkcolor=red, citecolor=cyan, urlcolor=magenta, bookmarks=false]{hyperref}
\usepackage[all]{hypcap}
\renewcommand*{\backref}[1]{[#1]}
\citestyle{apj}

%% wrap any troublesome environments in the following
%\capstartfalse 
%\capstarttrue


\newcommand{\m}[1]{\mathbf{#1}}
\newcommand{\ev}[1]{E\left\{{#1}\right\}}

\def\be{\begin{equation}}
\def\ee{\end{equation}}
\newcommand{\bb}{\begin{bmatrix}}
\newcommand{\eb}{\end{bmatrix}}
\newcommand{\lp}{\left(}
\newcommand{\rp}{\right)}


% Try to keep latex from splitting long footnotes
\interfootnotelinepenalty=100

\begin{document}

\title{Robust Pulsar Timing Inference with non-Gaussian Distributions}

\author{
J.~.A.~Ellis\altaffilmark{1,4},
H.~Tak\altaffilmark{2},
S.~K.~Ghosh\altaffilmark{3},
}

\affil{$^{1}$Jet Propulsion Laboratory, 4800 Oak Grove Drive, Pasadena, CA 91109, USA}
\affil{$^{2}$Statistical and Applied Mathematical Sciences Institute}
\affil{$^{3}$North Carolina State University}
\affil{$^{4}$Einstein Fellow}

\begin{abstract}
Really awesome results.
\end{abstract}

\section{Introduction}
\begin{itemize}
\item Overview of pulsar timing methodology.
\item Need for outlier modeling. Reference fact that several outliers are already found in current data and more non-gaussian errors are likely to crop up as data gets better.
\end{itemize}

\section{Methods}
\begin{itemize}
\item Give overview of mixture model and Gibbs setup. Maybe here we want to go over several different options such as one theta for all points vs a different theta for all points. Is there ever a case where we wouldn't want the d.o.f to vary?
\item Stress differences and improvements over \cite{vvh17}. Namely, our method makes fewer assumptions about the nature of the ``outliers'' given that the gaussian mixture case assumes that the ``outlier'' distributions is uniform (i.e. Gaussian with a very large variance) and we allow the data to inform the ``outlier'' distribution. Furthermore, even though we \emph{can} use this as an outlier rejection code, it is more aptly a robust inference code since the mixture is very flexible in our framework. Lastly, stress the computational advantages of this method. Can be coded with few extra lines over standard Gaussian likelihoods whereas \cite{vvh17} is an enormous effort with coordinate transformations, gradients, hessians, and customized samplers.
\end{itemize}

\section{Tests}
\begin{itemize}
\item Test illustrating method on simulated data. Something obvious to show that it can work with single large outliers or on a t-distributed dataset (i.e. many ``outliers")
\item Run on 9-year NANOGrav data (use this data because it is open) and show improvement over standard methods and maybe compare with \cite{vvh17} method. Need to have a metric for what we call ``better''. For some very precisely timed pulsars perhaps we could use the upper limit on the powerlaw red noise amplitude as a measure of ``goodness'' since that is a proxy for its sensitivity to a GW background.
\item Maybe test on Normal pulsar not a MSP. MSPs are what is used for GW experiments and are generally fairly well behaved, normal pulsars are much more messy and will be a good testing ground for robust techniques like this.
\end{itemize}

\acknowledgements

\emph{Acknowledgments.}
JAE acknowledges support by NASA through Einstein Fellowship grants PF4-150120.

\bibliographystyle{../style_files/apj}
\bibliography{ms}

\end{document}
